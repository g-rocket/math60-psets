\documentclass[boxes]{gsypset}

\mailbox{}
\class{Math 60}
\assignment{HW 3}
\duedate{May 19}

\begin{document}
	\begin{problem}[2.5.6]
		A rectangular stick of butter is placed in the microwave oven to melt. 
		When the butter's length is \SI{6}{in} and 
		its square cross section measures \SI{1.5}{in} on a side, 
		its length is decreasing at a rate of \SI{0.25}{in\per\min} and 
		its cross-sectional edge is decreasing at a rate of \SI{0.125}{in\per\min}. 
		How fast is the butter melting 
		(i.e., at what rate is the solid volume of butter turning to liquid) 
		at that instant?
	\end{problem}
	\begin{solution}
		
	\end{solution}
	
	\begin{problem}[2.5.14]
		Suppose that $z = f (x + y, x - y)$ has continuous partial derivatives with respect to 
		$u = x + y$ and $v = x - y$.
		Show that
		\[
			\pf{z}{x} \pf{z}{y} = \left(\pf{z}{u}\right)^2 -\ \left(\pf{z}{v}\right)^2
		\]
	\end{problem}
	\begin{solution}
		
	\end{solution}
	
	\begin{problem}[2.5.24]
		With
		\begin{align*}
			\mathbf{f}(x,y,z) &= (x^2y + y^2z, xyz, e^z), \\
			\mathbf{g}(t) &= (t-2, 3t+7, t^3),
		\end{align*}
		calculate $D(\mathbf{f} \circ \mathbf{g})$ in two ways:
		\begin{subproblems}
			\subproblem By first evaluating $\mathbf{f} \circ \mathbf{g}$
				\begin{solution}
					
				\end{solution}
				
			\subproblem By using the chain rule and the derivative matrices 
				$D\mathbf{f}$ and $D\mathbf{g}$.
				\begin{solution}
					
				\end{solution}
		\end{subproblems}
	\end{problem}
	
	\begin{problem}[2.5.36]
		Suppose that you are given an equation of the form
		\[
			F(x,y,z) = 0,
		\]
		for example, something like $x^3z + y\cos z + \frac{\sin y}{z} = 0$. 
		Then we may consider $z$ to be defined implicitly as a function $z(x, y)$.
		\begin{subproblems}
			\subproblem Use the chain rule to show that if 
				$F$ and $z(x,y)$ are both assumed to be differentiable, then
				\begin{align*}
					\pf{z}{x} &= -\frac{F_x(x,y,z)}{F_z(x,y,z)}, \\
					\pf{z}{y} &= -\frac{F_y(x,y,z)}{F_z(x,y,z)}
				\end{align*}
				\begin{solution}
					
				\end{solution}
				
			\subproblem Use part (a) to find $\pf{z}{x}$ and $\pf{z}{y}$
				where $z$ is given by the equation $xyz = 2$. 
				Check your result by explicitly solving for $z$
				and then calculating the partial derivatives.
				\begin{solution}
					
				\end{solution}
		\end{subproblems}
	\end{problem}
	
	\begin{problem}[2.6.6]
		Calculate the directional derivative of the given function $f$
		at the point $\mathbf{a}$ in the direction parallel to the vector $\mathbf{u}$.
		\begin{align*}
			f(x,y,z) &= xyz, &
			\mathbf{a} &= (-1,0,2), &
			\mathbf{u} &= \frac{2\mathbf{k} - \mathbf{i}}{\sqrt{5}}
		\end{align*}
	\end{problem}
	\begin{solution}
		
	\end{solution}
	
	\begin{problem}[2.6.12]
		A ladybug (who is very sensitive to temperature) is crawling on graph paper. 
		She is at the point $(3, 7)$ and notices that if she moves in the $\mathbf{i}$-direction,
		the temperature increases at a rate of \SI{3}{\deg\per\cm}. 
		If she moves in the $\mathbf{j}$-direction, she finds that 
		her temperature decreases at a rate of \SI{2}{\deg\per\cm}. 
		\begin{subproblems}
			\subproblem In what direction should the ladybug move if she wants to warm up most rapidly?
				\begin{solution}
					
				\end{solution}
				
			\subproblem In what direction should the ladybug move if she wants to cool off most rapidly?
				\begin{solution}
					
				\end{solution}
				
			\subproblem In what direction should the ladybug move if 
				she desires her temperature \textit{not} to change?
				\begin{solution}
					
				\end{solution}
		\end{subproblems}
	\end{problem}
	
	\begin{problem}[2.6.18]
		 Find an equation for the tangent plane to the surface given by the equation 
		 \[2xz+yz-x^2y+10=0\]
		 at the point $(x_0, y_0, z_0) = (1,-5,5)$.
	\end{problem}
	\begin{solution}
		
	\end{solution}
\end{document}