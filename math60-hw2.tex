\documentclass[oneperpage]{gsypset}

\name{}
\class{Math 60}
\assignment{HW 2}
\duedate{May 18, 2016}

\begin{document}
	\begin{problem}[2.3.24]
		Find the gradient $\nabla f(\mathbf{a})$ for
		\begin{align*}
			f(x,y,z) &= \cos z \ln(x + y^2) \\
			\mathbf{a} &= \left(e,0,\frac{\pi}{4}\right)
		\end{align*}
	\end{problem}
	\begin{solution}
		
	\end{solution}
	
	\begin{problem}[2.3.33]
		Find the matrix $D\mathbf{f}(\mathbf{a})$ of partial derivatives, for
		\begin{align*}
			\mathbf{f}(s,t) &= (s^2, st, t^2) \\
			\mathbf{a} &= (-1,1)
		\end{align*}
	\end{problem}
	\begin{solution}
		
	\end{solution}
	
	\begin{problem}[2.3.40]
		Find equations for the planes tangent to $z= x^2 - 6x + y^3$
		that are parallel to the plane $4x - 12y + z = 7$.
	\end{problem}
	\begin{solution}
		
	\end{solution}
	
	\begin{problem}[2.3.42]
		Suppose that you have the following information concerning a differentiable function $f$:
		\begin{align*}
			f(2,3) &= 12, &
			f(1.98,3) &= 12.1, &
			f(2,3.01) &= 12.2.
		\end{align*}
		\begin{subproblems}[(a)]
			\subproblem Give an approximate equation for the plane tangent to the graph of $f$ at (2,3,12).
			\begin{solution}
				
			\end{solution}
			
			\subproblem Use the result of part (a) to estimate $f(1.98, 2.98)$.
			\begin{solution}
				
			\end{solution}
		\end{subproblems}
	\end{problem}
	
	\begin{problem}[2.4.5]
		Verify the product and quotient rules (Proposition 4.2) for the pair of functions given below.
		\begin{align*}
			f(x,y) &= x^2y + y^3 \\
			g(x,y) &= \frac{x}{y}
		\end{align*}
		\fbox{\begin{minipage}{\linewidth}
			PROPOSITION 4.2:
			
			Let $f, g: X \subseteq \mathbb{R}^n \to \mathbb{R}$ be differentiable 
			at $\mathbf{a} \in X$. Then
			\begin{enumerate}[1.]
				\item The product function $f g$ is also differentiable at $\mathbf{a}$, and
					\[
						D(fg)(\mathbf{a}) = g(\mathbf{a})Df(\mathbf{a})+ f(\mathbf{a})Dg(\mathbf{a}).
					\]
				\item If $g(a) \ne 0$, then the quotient function $\frac{f}{g}$ is differentiable
					at $\mathbf{a}$, and
					\[
						D\left(\frac{f}{g}\right)(\mathbf{a})
							= \frac{g(\mathbf{a})Df(\mathbf{a}) - f(\mathbf{a})Dg(\mathbf{a})}{g(\mathbf{a})^2}
					\]
			\end{enumerate}
		\end{minipage}}
	\end{problem}
	\begin{solution}
		
	\end{solution}
	
	\begin{problem}[2.4.17]
		For the function given below determine all second-order partial derivatives
		(including mixed partials).
		\[
			f(x,y) = x^2e^y + e^{2z}
		\]
	\end{problem}
	\begin{solution}
		
	\end{solution}
	
	\begin{problem}[2.4.23]
		Let $f(x,y) = ye^{3x}$.
		Give general formulas for 
		$\frac{\partial^n f}{\partial x^n}$ and $\frac{\partial^n f}{\partial y^n}$
		where $n \geq 2$.
	\end{problem}
	\begin{solution}
		
	\end{solution}
	
	\begin{problem}[2.4.29a]
		The three-dimensional heat equation is the partial differential equation
		\[
			k \left(
					\frac{\partial^2 T}{\partial x^2} +
					\frac{\partial^2 T}{\partial y^2} +
					\frac{\partial^2 T}{\partial z^2}
				\right)
				= \frac{\partial T}{\partial t},
		\]
		where $k$ is a positive constant. 
		It models the temperature $T(x,y,z,t)$ at the point $(x,y,z)$ and time $t$ of a body in space.
		\begin{subproblems}[(a)]
			\subproblem
				We examine a simplified version of the heat equation. 
				Consider a straight wire ``coordinatized'' by $x$. 
				Then the temperature $T(x, t)$ at time $t$ and position $x$ along the wire is modeled by 
				the one-dimensional heat equation
				\[
					k \frac{\partial^2 T}{\partial x^2} = \frac{\partial T}{\partial t}
				\]
				Show that the function $T(x, t) = e^{-kt} \cos x$ satisfies this equation. 
				Note that if $t$ is held constant at value $t_0$, then $T(x , t_0)$ shows how 
				the temperature varies along the wire at time $t_0$. 
				Graph the curves $z = T(x,t_0)$ for $t_0 = \{0,1, 10\}$, 
				and use them to understand the graph of the surface $z = T (x , t)$ for $t \geq 0$. 
				Explain what happens to the temperature of the wire after a long period of time.
				\begin{solution}
					
				\end{solution}
		\end{subproblems}
	\end{problem}
\end{document}